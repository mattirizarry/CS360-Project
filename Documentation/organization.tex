\documentclass[11pt]{article}
\usepackage{graphicx} % This lets you include figures
\usepackage{hyperref} % This lets you make links to web locations
\usepackage[margin=0.5in]{geometry}
\usepackage[rightcaption]{sidecap}
\usepackage{subcaption}
\usepackage{wrapfig}
\usepackage{float}
\usepackage{imakeidx}
\usepackage{indentfirst}
\makeindex
%---------------------------Do Not Edit Anything Above This Line!!------------------------

% edit the line below, if needed, to change the directory name for your image files.
\graphicspath{ {./images/} }



\begin{document}

%---------------------------Edit Content in the Box to Create the Title Page--------------
\begin{titlepage}
   \begin{center}
       \vspace*{1cm}
	   \Huge
       \textbf{Star Wars}

       \vspace{0.5cm}
       \Large
       Sprint 1 \\
       September 7th, 2023 \\
   \end{center}

       \vspace{1.5cm}

\begin{table}[h!]
\centering
\begin{tabular}{|l|l|}
\hline
\textbf{Name} & \textbf{Email Address} \\ \hline
Matthew Irizarry         & matthew.irizarry745@topper.wku.edu         \\ \hline
Zach Vance         & zachary.vance141@topper.wku.edu         \\ \hline
Keimon Bush         & keimon.bush105@topper.wku.edu        \\ \hline
Jeremiah Harris         & jeremiah.harris978@topper.wku.edu         \\ \hline
\end{tabular}
\end{table}

%Latex Table Generator    
%https://www.tablesgenerator.com/     
        
\vspace{4in}

\centering        
CS 360 \\
Fall 2023\\
Project Organization Documentation

\end{titlepage}
%---------------------------Edit Content in the Box to Create the Title Page--------------


% No text here.


%---------------------------Do Not Edit Anything In This Box!!------------------------
%Table of contents and list of figures will be autogenerated by this section.
\newpage
\setcounter{page}{1}%
\cleardoublepage
\pagenumbering{gobble}
\tableofcontents
\cleardoublepage
\pagenumbering{arabic}
\clearpage
\newpage
\setcounter{page}{1}%
\cleardoublepage
\pagenumbering{gobble}
\listoffigures
\cleardoublepage
\pagenumbering{arabic}
\newpage
%---------------------------Do Not Edit Anything In This Box!!------------------------

% No text here.


%---------------------------Project Team's Organizational Approach------------------------------
\section{Project Team's Organizational Approach} %\section{} is used to create major section headers
%250 words minimum for each sprint
	\subsection{Sprint 1}%How/where did the group meet?  How often did you meet as an entire team?  Who’s the Project Manager for this sprint?

        For this sprint, Matthew Irizarry, was the project manager. We decided to meet twice a week after class, for approximately 30-45 minutes each time. We ended up going to the Commons to meet in person, as there was always somewhere for us to sit. We generally discussed quick progress updates, and wins to celebrate. However, we did recognize a few areas for improvement around scheduling and organizing documents into one place, but a solution was quickly implemented. Overall, the sprint was very efficient and effective, and allowed us to complete our work fully and in a timely manner. We plan to have a Sprint 1 Retrospective to learn from what we did, and improve on our process in the next sprint.
 
	\subsection{Sprint 2} %How/where did the group meet?  How often did you meet as an entire team?  Who’s the Project Manager for this sprint?

         For this sprint, Matthew Irizarry, was the project manager. We decided to meet, once again, twice a week after class, for approximately 30-45 minutes each time. During these meetings, we discussed how we have implemented the solutions we discussed during our sprint 1 retrospective, and if they are working to improve the organization issues that we encountered. Our overall consensus was that yes, GitHub did indeed help us stay more organized, and we were happy with how we quickly got onboarded to GitHub. Next, we would discuss how we were doing on tasks, and if there was anyone that needed help from other team members to complete their task.

	\subsection{Sprint 3}%How/where did the group meet?  How often did you meet as an entire team?  Who’s the Project Manager for this sprint?

        For this sprint, Matt Irizarry was the project manager. We pretty much followed the same meetings, however we saw the need for more once development of the game was underway. For these meetings, we would meet via Discord whenever two or three of us was available. This allowed everyone to stay mostly up to date on what was going on, but was not hindering development by missing one or two group members on a meeting. Overall, the sprint was a good sprint, and we learned a lot (especially merging and rebasing stuff on GitHub)

	\subsection{Sprint 4}%How/where did the group meet?  How often did you meet as an entire team?  Who’s the Project Manager for this sprint?



%---------------------------End Project Team's Organizational Approach------------------------------


% No text here.


%---------------------------Schedule Organization---------------------------------------------------
\section{Schedule Organization}
%Gantt charts cover the tasks/time commitments and estimations for the entire project.  We will have four iterations of the Gantt Chart, with iteration focusing on a specific sprint.

\subsection{Gantt Chart v1:}
%200 words minimum to describe the focus for this sprint.
%Identify the location for the Gantt Chart created during Sprint 1.  Should be clearly labeled in the project directory.

A Gantt chart is a project management tool that provides a detailed visual representation of a project's timeline, tasks, and progress. The Gantt chart displays the tasks that were worked on and completed during the first sprint. Tasks completed in the first sprint include: class readings, planning, task distribution, presentation, documentation, and CATME evaluations. These tasks aided the team in the planning and organization of the remainder of the project and allowed for the team to familiarize themselves with each other and work together better in future sprints.

The GANTT chart for this sprint has been included in the submission archive that was uploaded to blackboard.


\subsection{Gantt Chart v2:}
%200 words minimum to describe the focus for this sprint.
%Identify the location for the Gantt Chart created during Sprint 2.  Should be clearly labeled in the project directory.
For this sprint, we primarily focused on familiarizing ourselves with GitHub issues, commits and branching so we can hit the ground running in Sprint 3 once we start programming. However, the primary focus was our UML diagrams, the presentation, and the documentation for everything. Overall, we kept to a timely schedule, and everyone did really great work. The GANTT chart is included in the archive that is submitted to blackboard, as it was last time.


\subsection{Gantt Chart v3:}
%200 words minimum to describe the focus for this sprint.
%Identify the location for the Gantt Chart created during Sprint 3.  Should be clearly labeled in the project directory.

For this sprint, our primary tasks were making sure that we implemented everything as closely as possible to the way it was described in the Technical documentation. There were a few things that changed. For example, we no longer are going to use some kind of Node.js server, but instead are relying on Django for our backend. This gives us some relief with regards to security, as Django already comes with a robust security suite in place. Once we started this, we were quickly on our way to developing an MVP for our client. Finally, we had to make sure that we finished our documentation and CATME eval tasks. Everyone was very diligent and thorough with their assigned tasks.

GANTT chart is included in the archive that is submitted to blackboard, the same as it was last time.


\subsection{Final Gantt Chart:}
%200 words minimum to describe the focus for this sprint.
%Identify the location for the Gantt Chart created during Sprint 4.  Should be clearly labeled in the project directory.
Text goes here.



%---------------------------End Schedule Organization---------------------------------------------------


% No text here.


%---------------------------Progress Visibility---------------------------------------------------
\section{Progress Visibility}
%200 words minimum for each sprint.
%For each sprint, explain how each member of the group is progressing with assigned tasks and how that progress is shared with the group.  Also, explain how the group is progressing with assigned tasks and how that progress is shared with the client.  Examples:  how does the group assign tasks?  How to group members know tasks assigned to them?  How do group members communicate when assigned tasks are complete, need assistance, or waiting on other tasks to be completed first?
\subsection{Sprint 1 Progress Visibility}

For this sprint, we focused on establishing channels of communication, meeting times that worked for everyone, and just generally got to know each other. We successfully divided up the tasks for the technical documentation, the presentation, and the organizational documentation. After that, we got straight to work. There was never any point where we had to remind individuals to stay on task, but I did make sure to give reminders of what was upcoming. Everyone was able to use these quick reminders to stay on task and get their work done, which was really nice. We used our primary channel of communication, Discord, to handle any situations that we could over text, and whenever necessary would utilize Discord or an in person meeting to clarify things that couldn't be clarified over text. Both meetings with Galloway Games went well, and we were able to use these meetings to dive deeper where necessary, and pivot away from work that wasn't within the scope of this sprint, or the project whatsoever.

\subsection{Sprint 2 Progress Visibility}

For this sprint, we focused on GitHub being the source of progress tracking for this repository. For each line item on the rubric, it was divided up and organized under the epic < story < feature paradigm, and then team members self-assigned issues depending on what was due next. In addition to this, the issues had all the information necessary to complete, and thus it was very easy to start work on the sprint this time around. I, Matt, was responsible for making sure everyone had enough tasks where they didn't feel overwhelmed but were making significant contributions to the team as a whole. 

When we were able to meet with Dr. Galloway, it was really easy for us to know where we stood and thus it was easy to tell him what we had accomplished so far. Overall we were really happy with the changes that we implemented this sprint, and I am sure we will discuss it more in our retrospective for this sprint.

\subsection{Sprint 3 Progress Visibility}

In Sprint 3 tasks were assigned primarily via volunteering and task lists are displayed on both GitHub and Discord. Team members have each created respective branches on GitHub which they upload their work to.The progress is periodically merged and checked for conflicts. The team regularly communicated via discord calls and text messages, particularly as deliverables were due. Pair programming has taken place on discord to help encourage collaboration and help teammates problem solve when issues arise during the development process. The Star Wars game has progressed well and the team has established a solid base for progress to be accomplished throughout Sprint 4 and finish strong. Progress has been shared with the client via weekly team-client meetings and a demo which was delivered via a presentation. The demo showcased our level design and player movement. Some enemy design has taken place, but did not make it into the demo. Additionally, a Django Database has been implemented and is functional, but does not yet interact with the game. The user interface has not yet been completed but the design has been planned. Overall, the team has made good progress and is collaborating well. Looking forward to sprint 4, the team has well defined goals regarding game design, performance, and documentation which will make it easier for the software to be successfully developed.

\subsection{Sprint 4 Progress Visibility}
Text goes here.

%---------------------------End Progress Visibility---------------------------------------------------

% No text here.


%---------------------------Software Process Model---------------------------------------------------
\section{Software Process Model}
%200 words minimum
%Describe in this section the Software Process Model used and how it increases the quality of the final deliverables.  The team should also define the quality control steps that are used in the Software Process Model.

For this project, we find it paramount that we be able to effectively plan in the long term, but also be able to make changes on the fly if requested by the client, or if it is demanded by a change in software or something out of our control. As such, we will be adopting the Agile organizational model. Comparing this to something like the waterfall approach, which is a linear software development model, agile allows a team to make iterative changes over several sprints. This allows for better pivoting when necessary, and encourages small iterative development over large swaths of changes all at once. Doing smaller, iterative changes like this allows bugs to be caught earlier, and squashed with less effort. This will have a significant impact in the long run, not just in the quality of the software, but especially in the amount of time that will be saved from this iterative approach. 

%---------------------------End Software Process Model---------------------------------------------------

% No text here.


%---------------------------Risk Management--------------------------------------------------------------
\section{Risk Management}
%Use this section to describe the team's approach to risk management.

\subsection{Risk Identification}
%List, categorize, and prioritize all potential risks associated with the project.

The understanding of when a risk occurs can be crucial to making progress as a team in developing software. This is because if you let a risk go unnoticed it can cause other problems to occur beyond the initial risk and something that was once small is now a burden to the entire team and all resources and time are now committed to solving these other problems instead of progressing in development. The key solution to identifying risks is tight communication between members whether it is through code review, communication schedules, or setting up meetings. More ways in identifying risks is to not look over the small things that appear. If something small happens it is best to write it down for later discussion as issues most of the time start small but grow in size. As well as identifying how bad an unexpected event will be on the project. If we can identify the risk an issue poses on the project we can more easily combat said issue.



\subsection{Risk Planning}
%Give overviews of plans for specific risk types

Risk planning is the process of looking in the future and examining possible risks because if you can identify a risk before it comes you can plan your workflow around said risk to minimize any conflict. You can separate these risks into 3 categories: Known Knowns, Known Unknowns, and Unknown Unknowns. The first 2 options are more feasible to plan around as the team knows to some degree what and/or when to expect the risks. However, the most dangerous is Unknown Unknown as there is no telling when it will hit your team or what the exact issue will be. For unknown unknowns these are ones we take as we see them. As they can not be planned for and generally happen when least expected we can perform damage control. This will happen by a discussion of the situation and how to tackle said issue. Depending on what the group feels is the best course for action will depend on how these are handled. For the known knowns we can easily plan for these and have predetermined steps to fix said issues when they arise as these are expected issues. For the known unknowns we can’t have a set plan for these as even though we know of them we do not know if they will arise and when. For this though we have more basic plans of being ahead. So if someone unexpectedly becomes sick we have some plans that we discuss with the group quickly and take the best course of action.

\subsection{Risk Monitoring}
%Give overviews of how the team will monitor specific types of risks

Monitoring the risk is quite simple with the software architecture that the team will implement. GitHub has an ‘Issues’ tab that we can use to keep track of any problems software-wise that can happen where teammates can comment on the issue and work together to resolve the risk. Team dynamics risks such as the possibility of losing a member, scheduling issues, or disagreement can be solved with meetings and double-checking that work is split fairly amongst members so that tasks can be done as efficiently as possible. As well as having a check up where we ask all members how they feel about current situations. We do these checks at most every meeting as it is a great way to get an idea of how everyone is feeling at the current time. If someone says something that they believe to be a risk it is also taken into account and depending on the situation we will have an immediate discussion or a discussion for later. This can help with us solving said issues that appear and seeing if a risk is actually a real risk or a momentary issue.


%---------------------------Risk Management--------------------------------------------------------------





%example image:  uncomment to show usage
%\begin{figure}[h]
%    \centering
%    \includegraphics[width=1\textwidth]{images/Add_non-music.png}
%    \caption{This is how you add non-music items.}
%    \label{fig16}
%\end{figure}


%example links:  uncomment to show usage.
%\url{https://www.youtube.com}
%\href{https://www.wku.edu/}{WKU Homepage}
%\footnote{You can put the link in a footnote like this.}

% Anything to the right of a percent sign will be ignored by LaTeX.
% You can use this to put notes to yourself.  



\end{document}
